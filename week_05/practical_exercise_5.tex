% Options for packages loaded elsewhere
\PassOptionsToPackage{unicode}{hyperref}
\PassOptionsToPackage{hyphens}{url}
%
\documentclass[
]{article}
\title{practical\_exercise\_5, Methods 3, 2021, autumn semester}
\author{Otto Sejrskild Santesson}
\date{13th October 2021}

\usepackage{amsmath,amssymb}
\usepackage{lmodern}
\usepackage{iftex}
\ifPDFTeX
  \usepackage[T1]{fontenc}
  \usepackage[utf8]{inputenc}
  \usepackage{textcomp} % provide euro and other symbols
\else % if luatex or xetex
  \usepackage{unicode-math}
  \defaultfontfeatures{Scale=MatchLowercase}
  \defaultfontfeatures[\rmfamily]{Ligatures=TeX,Scale=1}
\fi
% Use upquote if available, for straight quotes in verbatim environments
\IfFileExists{upquote.sty}{\usepackage{upquote}}{}
\IfFileExists{microtype.sty}{% use microtype if available
  \usepackage[]{microtype}
  \UseMicrotypeSet[protrusion]{basicmath} % disable protrusion for tt fonts
}{}
\makeatletter
\@ifundefined{KOMAClassName}{% if non-KOMA class
  \IfFileExists{parskip.sty}{%
    \usepackage{parskip}
  }{% else
    \setlength{\parindent}{0pt}
    \setlength{\parskip}{6pt plus 2pt minus 1pt}}
}{% if KOMA class
  \KOMAoptions{parskip=half}}
\makeatother
\usepackage{xcolor}
\IfFileExists{xurl.sty}{\usepackage{xurl}}{} % add URL line breaks if available
\IfFileExists{bookmark.sty}{\usepackage{bookmark}}{\usepackage{hyperref}}
\hypersetup{
  pdftitle={practical\_exercise\_5, Methods 3, 2021, autumn semester},
  pdfauthor={Otto Sejrskild Santesson},
  hidelinks,
  pdfcreator={LaTeX via pandoc}}
\urlstyle{same} % disable monospaced font for URLs
\usepackage[margin=1in]{geometry}
\usepackage{color}
\usepackage{fancyvrb}
\newcommand{\VerbBar}{|}
\newcommand{\VERB}{\Verb[commandchars=\\\{\}]}
\DefineVerbatimEnvironment{Highlighting}{Verbatim}{commandchars=\\\{\}}
% Add ',fontsize=\small' for more characters per line
\usepackage{framed}
\definecolor{shadecolor}{RGB}{248,248,248}
\newenvironment{Shaded}{\begin{snugshade}}{\end{snugshade}}
\newcommand{\AlertTok}[1]{\textcolor[rgb]{0.94,0.16,0.16}{#1}}
\newcommand{\AnnotationTok}[1]{\textcolor[rgb]{0.56,0.35,0.01}{\textbf{\textit{#1}}}}
\newcommand{\AttributeTok}[1]{\textcolor[rgb]{0.77,0.63,0.00}{#1}}
\newcommand{\BaseNTok}[1]{\textcolor[rgb]{0.00,0.00,0.81}{#1}}
\newcommand{\BuiltInTok}[1]{#1}
\newcommand{\CharTok}[1]{\textcolor[rgb]{0.31,0.60,0.02}{#1}}
\newcommand{\CommentTok}[1]{\textcolor[rgb]{0.56,0.35,0.01}{\textit{#1}}}
\newcommand{\CommentVarTok}[1]{\textcolor[rgb]{0.56,0.35,0.01}{\textbf{\textit{#1}}}}
\newcommand{\ConstantTok}[1]{\textcolor[rgb]{0.00,0.00,0.00}{#1}}
\newcommand{\ControlFlowTok}[1]{\textcolor[rgb]{0.13,0.29,0.53}{\textbf{#1}}}
\newcommand{\DataTypeTok}[1]{\textcolor[rgb]{0.13,0.29,0.53}{#1}}
\newcommand{\DecValTok}[1]{\textcolor[rgb]{0.00,0.00,0.81}{#1}}
\newcommand{\DocumentationTok}[1]{\textcolor[rgb]{0.56,0.35,0.01}{\textbf{\textit{#1}}}}
\newcommand{\ErrorTok}[1]{\textcolor[rgb]{0.64,0.00,0.00}{\textbf{#1}}}
\newcommand{\ExtensionTok}[1]{#1}
\newcommand{\FloatTok}[1]{\textcolor[rgb]{0.00,0.00,0.81}{#1}}
\newcommand{\FunctionTok}[1]{\textcolor[rgb]{0.00,0.00,0.00}{#1}}
\newcommand{\ImportTok}[1]{#1}
\newcommand{\InformationTok}[1]{\textcolor[rgb]{0.56,0.35,0.01}{\textbf{\textit{#1}}}}
\newcommand{\KeywordTok}[1]{\textcolor[rgb]{0.13,0.29,0.53}{\textbf{#1}}}
\newcommand{\NormalTok}[1]{#1}
\newcommand{\OperatorTok}[1]{\textcolor[rgb]{0.81,0.36,0.00}{\textbf{#1}}}
\newcommand{\OtherTok}[1]{\textcolor[rgb]{0.56,0.35,0.01}{#1}}
\newcommand{\PreprocessorTok}[1]{\textcolor[rgb]{0.56,0.35,0.01}{\textit{#1}}}
\newcommand{\RegionMarkerTok}[1]{#1}
\newcommand{\SpecialCharTok}[1]{\textcolor[rgb]{0.00,0.00,0.00}{#1}}
\newcommand{\SpecialStringTok}[1]{\textcolor[rgb]{0.31,0.60,0.02}{#1}}
\newcommand{\StringTok}[1]{\textcolor[rgb]{0.31,0.60,0.02}{#1}}
\newcommand{\VariableTok}[1]{\textcolor[rgb]{0.00,0.00,0.00}{#1}}
\newcommand{\VerbatimStringTok}[1]{\textcolor[rgb]{0.31,0.60,0.02}{#1}}
\newcommand{\WarningTok}[1]{\textcolor[rgb]{0.56,0.35,0.01}{\textbf{\textit{#1}}}}
\usepackage{graphicx}
\makeatletter
\def\maxwidth{\ifdim\Gin@nat@width>\linewidth\linewidth\else\Gin@nat@width\fi}
\def\maxheight{\ifdim\Gin@nat@height>\textheight\textheight\else\Gin@nat@height\fi}
\makeatother
% Scale images if necessary, so that they will not overflow the page
% margins by default, and it is still possible to overwrite the defaults
% using explicit options in \includegraphics[width, height, ...]{}
\setkeys{Gin}{width=\maxwidth,height=\maxheight,keepaspectratio}
% Set default figure placement to htbp
\makeatletter
\def\fps@figure{htbp}
\makeatother
\setlength{\emergencystretch}{3em} % prevent overfull lines
\providecommand{\tightlist}{%
  \setlength{\itemsep}{0pt}\setlength{\parskip}{0pt}}
\setcounter{secnumdepth}{-\maxdimen} % remove section numbering
\ifLuaTeX
  \usepackage{selnolig}  % disable illegal ligatures
\fi

\begin{document}
\maketitle

\hypertarget{exercises-and-objectives}{%
\section{Exercises and objectives}\label{exercises-and-objectives}}

The objectives of the exercises of this assignment are based on:
\url{https://doi.org/10.1016/j.concog.2019.03.007}

\begin{enumerate}
\def\labelenumi{\arabic{enumi})}
\setcounter{enumi}{3}
\tightlist
\item
  Download and organise the data from experiment 1\\
\item
  Use log-likelihood ratio tests to evaluate logistic regression
  models\\
\item
  Test linear hypotheses\\
\item
  Estimate psychometric functions for the Perceptual Awareness Scale and
  evaluate them
\end{enumerate}

REMEMBER: In your report, make sure to include code that can reproduce
the answers requested in the exercises below (\textbf{MAKE A KNITTED
VERSION})\\
REMEMBER: This is part 2 of Assignment 2 and will be part of your final
portfolio

\hypertarget{exercise-4---download-and-organise-the-data-from-experiment-1}{%
\section{EXERCISE 4 - Download and organise the data from experiment
1}\label{exercise-4---download-and-organise-the-data-from-experiment-1}}

Go to \url{https://osf.io/ecxsj/files/} and download the files
associated with Experiment 1 (there should be 29).\\
The data is associated with Experiment 1 of the article at the following
DOI \url{https://doi.org/10.1016/j.concog.2019.03.007}

\begin{enumerate}
\def\labelenumi{\arabic{enumi})}
\tightlist
\item
  Put the data from all subjects into a single data frame - note that
  some of the subjects do not have the \emph{seed} variable. For these
  subjects, add this variable and make in \emph{NA} for all
  observations. (The \emph{seed} variable will not be part of the
  analysis and is not an experimental variable)
\end{enumerate}

\begin{Shaded}
\begin{Highlighting}[]
\NormalTok{df }\OtherTok{=} \FunctionTok{read\_bulk}\NormalTok{(}
  \AttributeTok{directory=} \StringTok{"experiment\_1/"}\NormalTok{,}\CommentTok{\# "Audition/" is the foldername where data is stored}
  \AttributeTok{fun =}\NormalTok{ read\_csv)}
\end{Highlighting}
\end{Shaded}

\begin{enumerate}
\def\labelenumi{\roman{enumi}.}
\tightlist
\item
  Factorise the variables that need factorising
\end{enumerate}

\begin{Shaded}
\begin{Highlighting}[]
\FunctionTok{glimpse}\NormalTok{(df)}
\end{Highlighting}
\end{Shaded}

\begin{verbatim}
## Rows: 25,602
## Columns: 18
## $ trial.type      <chr> "practice", "practice", "practice", "practice", "pract~
## $ pas             <dbl> 4, 4, 3, 2, 3, 2, 1, 1, 1, 4, 3, 4, 3, 3, 3, 1, 1, 1, ~
## $ trial           <dbl> 0, 1, 2, 3, 4, 5, 6, 7, 8, 9, 10, 11, 12, 13, 14, 15, ~
## $ jitter.x        <dbl> -0.43065222, -0.29154764, 0.30826178, 0.23776194, -0.4~
## $ jitter.y        <dbl> -0.33548153, -0.18243930, 0.40576314, 0.17474446, -0.2~
## $ odd.digit       <dbl> 9, 5, 5, 9, 3, 9, 3, 5, 7, 9, 7, 9, 5, 3, 7, 7, 3, 3, ~
## $ target.contrast <dbl> 0.1, 0.1, 0.1, 0.1, 0.1, 0.1, 0.1, 0.1, 0.1, 0.1, 0.1,~
## $ target.frames   <dbl> 9, 8, 7, 6, 5, 4, 3, 2, 1, 9, 8, 7, 6, 5, 4, 3, 2, 1, ~
## $ cue             <dbl> 0, 0, 0, 0, 0, 0, 0, 0, 0, 0, 0, 0, 0, 0, 0, 0, 0, 0, ~
## $ task            <chr> "quadruplet", "quadruplet", "quadruplet", "quadruplet"~
## $ target.type     <chr> "even", "odd", "even", "odd", "even", "odd", "even", "~
## $ rt.subj         <dbl> 4.9348001, 8.6709449, 8.4068649, 2.1268780, 1.8229408,~
## $ rt.obj          <dbl> 1.4180470, 0.8598070, 0.7461591, 1.6758869, 0.8470879,~
## $ even.digit      <dbl> 8, 4, 2, 8, 6, 6, 6, 4, 4, 8, 2, 2, 4, 2, 4, 6, 2, 8, ~
## $ seed            <dbl> 93764, 93764, 93764, 93764, 93764, 93764, 93764, 93764~
## $ obj.resp        <chr> "e", "o", "e", "o", "e", "o", "o", "e", "e", "o", "e",~
## $ subject         <chr> "001", "001", "001", "001", "001", "001", "001", "001"~
## $ File            <chr> "001.csv", "001.csv", "001.csv", "001.csv", "001.csv",~
\end{verbatim}

\begin{Shaded}
\begin{Highlighting}[]
\FunctionTok{ls.str}\NormalTok{(df)}
\end{Highlighting}
\end{Shaded}

\begin{verbatim}
## cue :  num [1:25602] 0 0 0 0 0 0 0 0 0 0 ...
## even.digit :  num [1:25602] 8 4 2 8 6 6 6 4 4 8 ...
## File :  chr [1:25602] "001.csv" "001.csv" "001.csv" "001.csv" "001.csv" "001.csv" ...
## jitter.x :  num [1:25602] -0.431 -0.292 0.308 0.238 -0.427 ...
## jitter.y :  num [1:25602] -0.335 -0.182 0.406 0.175 -0.221 ...
## obj.resp :  chr [1:25602] "e" "o" "e" "o" "e" "o" "o" "e" "e" "o" "e" "o" "e" "o" "e" ...
## odd.digit :  num [1:25602] 9 5 5 9 3 9 3 5 7 9 ...
## pas :  num [1:25602] 4 4 3 2 3 2 1 1 1 4 ...
## rt.obj :  num [1:25602] 1.418 0.86 0.746 1.676 0.847 ...
## rt.subj :  num [1:25602] 4.93 8.67 8.41 2.13 1.82 ...
## seed :  num [1:25602] 93764 93764 93764 93764 93764 ...
## subject :  chr [1:25602] "001" "001" "001" "001" "001" "001" "001" "001" "001" "001" ...
## target.contrast :  num [1:25602] 0.1 0.1 0.1 0.1 0.1 0.1 0.1 0.1 0.1 0.1 ...
## target.frames :  num [1:25602] 9 8 7 6 5 4 3 2 1 9 ...
## target.type :  chr [1:25602] "even" "odd" "even" "odd" "even" "odd" "even" "odd" "even" ...
## task :  chr [1:25602] "quadruplet" "quadruplet" "quadruplet" "quadruplet" ...
## trial :  num [1:25602] 0 1 2 3 4 5 6 7 8 9 ...
## trial.type :  chr [1:25602] "practice" "practice" "practice" "practice" "practice" ...
\end{verbatim}

\begin{Shaded}
\begin{Highlighting}[]
\NormalTok{df}\SpecialCharTok{$}\NormalTok{trial.type }\OtherTok{=} \FunctionTok{as.factor}\NormalTok{(df}\SpecialCharTok{$}\NormalTok{trial.type)}
\NormalTok{df}\SpecialCharTok{$}\NormalTok{pas }\OtherTok{=} \FunctionTok{as.factor}\NormalTok{(df}\SpecialCharTok{$}\NormalTok{pas)}
\NormalTok{df}\SpecialCharTok{$}\NormalTok{trial }\OtherTok{=} \FunctionTok{as.factor}\NormalTok{(df}\SpecialCharTok{$}\NormalTok{trial)}
\NormalTok{df}\SpecialCharTok{$}\NormalTok{odd.digit }\OtherTok{=} \FunctionTok{as.factor}\NormalTok{(df}\SpecialCharTok{$}\NormalTok{odd.digit)}
\NormalTok{df}\SpecialCharTok{$}\NormalTok{target.contrast }\OtherTok{=} \FunctionTok{as.factor}\NormalTok{(df}\SpecialCharTok{$}\NormalTok{target.contrast)}
\NormalTok{df}\SpecialCharTok{$}\NormalTok{cue }\OtherTok{=} \FunctionTok{as.factor}\NormalTok{(df}\SpecialCharTok{$}\NormalTok{cue)}
\NormalTok{df}\SpecialCharTok{$}\NormalTok{task }\OtherTok{=} \FunctionTok{as.factor}\NormalTok{(df}\SpecialCharTok{$}\NormalTok{task)}
\NormalTok{df}\SpecialCharTok{$}\NormalTok{target.type }\OtherTok{=} \FunctionTok{as.factor}\NormalTok{(df}\SpecialCharTok{$}\NormalTok{target.type)}
\NormalTok{df}\SpecialCharTok{$}\NormalTok{even.digit }\OtherTok{=} \FunctionTok{as.factor}\NormalTok{(df}\SpecialCharTok{$}\NormalTok{even.digit)}
\NormalTok{df}\SpecialCharTok{$}\NormalTok{seed }\OtherTok{=} \FunctionTok{as.factor}\NormalTok{(df}\SpecialCharTok{$}\NormalTok{seed)}
\NormalTok{df}\SpecialCharTok{$}\NormalTok{obj.resp }\OtherTok{=} \FunctionTok{as.factor}\NormalTok{(df}\SpecialCharTok{$}\NormalTok{obj.resp)}
\NormalTok{df}\SpecialCharTok{$}\NormalTok{subject }\OtherTok{=} \FunctionTok{as.factor}\NormalTok{(df}\SpecialCharTok{$}\NormalTok{subject)}
\end{Highlighting}
\end{Shaded}

\begin{enumerate}
\def\labelenumi{\roman{enumi}.}
\setcounter{enumi}{1}
\tightlist
\item
  Remove the practice trials from the dataset (see the \emph{trial.type}
  variable)
\end{enumerate}

\begin{Shaded}
\begin{Highlighting}[]
\FunctionTok{levels}\NormalTok{(df}\SpecialCharTok{$}\NormalTok{trial.type)}
\end{Highlighting}
\end{Shaded}

\begin{verbatim}
## [1] "experiment" "practice"
\end{verbatim}

\begin{Shaded}
\begin{Highlighting}[]
\NormalTok{df }\OtherTok{=}\NormalTok{ df }\SpecialCharTok{\%\textgreater{}\%} 
  \FunctionTok{filter}\NormalTok{(trial.type }\SpecialCharTok{!=} \StringTok{"practice"}\NormalTok{)}
\FunctionTok{view}\NormalTok{(df)}
\end{Highlighting}
\end{Shaded}

\begin{enumerate}
\def\labelenumi{\roman{enumi}.}
\setcounter{enumi}{2}
\tightlist
\item
  Create a \emph{correct} variable
\end{enumerate}

\begin{Shaded}
\begin{Highlighting}[]
\NormalTok{df }\OtherTok{=}\NormalTok{ df }\SpecialCharTok{\%\textgreater{}\%} 
  \FunctionTok{mutate}\NormalTok{(}\AttributeTok{correct =} \FunctionTok{ifelse}\NormalTok{(target.type }\SpecialCharTok{==} \StringTok{"odd"} \SpecialCharTok{\&}\NormalTok{ obj.resp }\SpecialCharTok{==} \StringTok{"o"}\NormalTok{,}\DecValTok{1}\NormalTok{,}
                          \FunctionTok{ifelse}\NormalTok{(target.type }\SpecialCharTok{==} \StringTok{"even"} \SpecialCharTok{\&}\NormalTok{ obj.resp }\SpecialCharTok{==} \StringTok{"e"}\NormalTok{,}\DecValTok{1}\NormalTok{,}\DecValTok{0}\NormalTok{)))}

\NormalTok{df}\SpecialCharTok{$}\NormalTok{correct }\OtherTok{=} \FunctionTok{as.logical}\NormalTok{(df}\SpecialCharTok{$}\NormalTok{correct)}
\end{Highlighting}
\end{Shaded}

\begin{enumerate}
\def\labelenumi{\roman{enumi}.}
\setcounter{enumi}{3}
\tightlist
\item
  Describe how the \emph{target.contrast} and \emph{target.frames}
  variables differ compared to the data from part 1 of this assignment
\end{enumerate}

In experiment 1, each participant had the same contrast level on the cue
whereas in experiment 2, the staircase round created an individual
contrast level for each participant. This is also what we see in the
data

As for target.frames, in experiment two, all participants saw the target
frame for 3 frames, whereas in experiment 1, participants saw the target
for 1-6 frames, equally distributed over participants.

\hypertarget{exercise-5---use-log-likelihood-ratio-tests-to-evaluate-logistic-regression-models}{%
\section{EXERCISE 5 - Use log-likelihood ratio tests to evaluate
logistic regression
models}\label{exercise-5---use-log-likelihood-ratio-tests-to-evaluate-logistic-regression-models}}

\begin{enumerate}
\def\labelenumi{\arabic{enumi})}
\tightlist
\item
  Do logistic regression - \emph{correct} as the dependent variable and
  \emph{target.frames} as the independent variable. (Make sure that you
  understand what \emph{target.frames} encode). Create two models - a
  pooled model and a partial-pooling model. The partial-pooling model
  should include a subject-specific intercept.
\end{enumerate}

\begin{Shaded}
\begin{Highlighting}[]
\NormalTok{log\_model\_pool }\OtherTok{=} \FunctionTok{glm}\NormalTok{(correct }\SpecialCharTok{\textasciitilde{}}\NormalTok{ target.frames, }\AttributeTok{data =}\NormalTok{ df, }\AttributeTok{family =} \FunctionTok{binomial}\NormalTok{(}\AttributeTok{link =} \StringTok{"logit"}\NormalTok{))}

\NormalTok{log\_model\_partial\_pool }\OtherTok{=} \FunctionTok{glmer}\NormalTok{(correct }\SpecialCharTok{\textasciitilde{}}\NormalTok{ target.frames }\SpecialCharTok{+}\NormalTok{ (}\DecValTok{1}\SpecialCharTok{|}\NormalTok{subject), }\AttributeTok{data =}\NormalTok{ df, }\AttributeTok{family =} \FunctionTok{binomial}\NormalTok{(}\AttributeTok{link =} \StringTok{"logit"}\NormalTok{))}
\end{Highlighting}
\end{Shaded}

\begin{Shaded}
\begin{Highlighting}[]
\NormalTok{df\_2 }\OtherTok{=} \FunctionTok{read\_bulk}\NormalTok{(}
  \AttributeTok{directory=} \StringTok{"experiment\_1/"}\NormalTok{,}
  \AttributeTok{fun =}\NormalTok{ read\_csv)}
\end{Highlighting}
\end{Shaded}

\begin{verbatim}
## Reading 001.csv
\end{verbatim}

\begin{verbatim}
## Rows: 882 Columns: 17
\end{verbatim}

\begin{verbatim}
## -- Column specification --------------------------------------------------------
## Delimiter: ","
## chr  (5): trial.type, task, target.type, obj.resp, subject
## dbl (12): pas, trial, jitter.x, jitter.y, odd.digit, target.contrast, target...
\end{verbatim}

\begin{verbatim}
## 
## i Use `spec()` to retrieve the full column specification for this data.
## i Specify the column types or set `show_col_types = FALSE` to quiet this message.
\end{verbatim}

\begin{verbatim}
## Reading 002.csv
\end{verbatim}

\begin{verbatim}
## Rows: 882 Columns: 17
\end{verbatim}

\begin{verbatim}
## -- Column specification --------------------------------------------------------
## Delimiter: ","
## chr  (5): trial.type, task, target.type, obj.resp, subject
## dbl (12): pas, trial, jitter.x, jitter.y, odd.digit, target.contrast, target...
\end{verbatim}

\begin{verbatim}
## 
## i Use `spec()` to retrieve the full column specification for this data.
## i Specify the column types or set `show_col_types = FALSE` to quiet this message.
\end{verbatim}

\begin{verbatim}
## Reading 003.csv
\end{verbatim}

\begin{verbatim}
## Rows: 882 Columns: 17
\end{verbatim}

\begin{verbatim}
## -- Column specification --------------------------------------------------------
## Delimiter: ","
## chr  (5): trial.type, task, target.type, obj.resp, subject
## dbl (12): pas, trial, jitter.x, jitter.y, odd.digit, target.contrast, target...
\end{verbatim}

\begin{verbatim}
## 
## i Use `spec()` to retrieve the full column specification for this data.
## i Specify the column types or set `show_col_types = FALSE` to quiet this message.
\end{verbatim}

\begin{verbatim}
## Reading 004.csv
\end{verbatim}

\begin{verbatim}
## Rows: 882 Columns: 17
\end{verbatim}

\begin{verbatim}
## -- Column specification --------------------------------------------------------
## Delimiter: ","
## chr  (5): trial.type, task, target.type, obj.resp, subject
## dbl (12): pas, trial, jitter.x, jitter.y, odd.digit, target.contrast, target...
\end{verbatim}

\begin{verbatim}
## 
## i Use `spec()` to retrieve the full column specification for this data.
## i Specify the column types or set `show_col_types = FALSE` to quiet this message.
\end{verbatim}

\begin{verbatim}
## Reading 005.csv
\end{verbatim}

\begin{verbatim}
## Rows: 882 Columns: 17
\end{verbatim}

\begin{verbatim}
## -- Column specification --------------------------------------------------------
## Delimiter: ","
## chr  (5): trial.type, task, target.type, obj.resp, subject
## dbl (12): pas, trial, jitter.x, jitter.y, odd.digit, target.contrast, target...
\end{verbatim}

\begin{verbatim}
## 
## i Use `spec()` to retrieve the full column specification for this data.
## i Specify the column types or set `show_col_types = FALSE` to quiet this message.
\end{verbatim}

\begin{verbatim}
## Reading 006.csv
\end{verbatim}

\begin{verbatim}
## Rows: 882 Columns: 17
\end{verbatim}

\begin{verbatim}
## -- Column specification --------------------------------------------------------
## Delimiter: ","
## chr  (5): trial.type, task, target.type, obj.resp, subject
## dbl (12): pas, trial, jitter.x, jitter.y, odd.digit, target.contrast, target...
\end{verbatim}

\begin{verbatim}
## 
## i Use `spec()` to retrieve the full column specification for this data.
## i Specify the column types or set `show_col_types = FALSE` to quiet this message.
\end{verbatim}

\begin{verbatim}
## Reading 007.csv
\end{verbatim}

\begin{verbatim}
## Rows: 882 Columns: 16
\end{verbatim}

\begin{verbatim}
## -- Column specification --------------------------------------------------------
## Delimiter: ","
## chr  (5): trial.type, task, target.type, obj.resp, subject
## dbl (11): pas, trial, jitter.x, jitter.y, odd.digit, target.contrast, target...
\end{verbatim}

\begin{verbatim}
## 
## i Use `spec()` to retrieve the full column specification for this data.
## i Specify the column types or set `show_col_types = FALSE` to quiet this message.
\end{verbatim}

\begin{verbatim}
## Reading 008.csv
\end{verbatim}

\begin{verbatim}
## Rows: 882 Columns: 17
\end{verbatim}

\begin{verbatim}
## -- Column specification --------------------------------------------------------
## Delimiter: ","
## chr  (5): trial.type, task, target.type, obj.resp, subject
## dbl (12): pas, trial, jitter.x, jitter.y, odd.digit, target.contrast, target...
\end{verbatim}

\begin{verbatim}
## 
## i Use `spec()` to retrieve the full column specification for this data.
## i Specify the column types or set `show_col_types = FALSE` to quiet this message.
\end{verbatim}

\begin{verbatim}
## Reading 009.csv
\end{verbatim}

\begin{verbatim}
## Rows: 882 Columns: 17
\end{verbatim}

\begin{verbatim}
## -- Column specification --------------------------------------------------------
## Delimiter: ","
## chr  (5): trial.type, task, target.type, obj.resp, subject
## dbl (12): pas, trial, jitter.x, jitter.y, odd.digit, target.contrast, target...
\end{verbatim}

\begin{verbatim}
## 
## i Use `spec()` to retrieve the full column specification for this data.
## i Specify the column types or set `show_col_types = FALSE` to quiet this message.
\end{verbatim}

\begin{verbatim}
## Reading 010.csv
\end{verbatim}

\begin{verbatim}
## Rows: 882 Columns: 17
\end{verbatim}

\begin{verbatim}
## -- Column specification --------------------------------------------------------
## Delimiter: ","
## chr  (5): trial.type, task, target.type, obj.resp, subject
## dbl (12): pas, trial, jitter.x, jitter.y, odd.digit, target.contrast, target...
\end{verbatim}

\begin{verbatim}
## 
## i Use `spec()` to retrieve the full column specification for this data.
## i Specify the column types or set `show_col_types = FALSE` to quiet this message.
\end{verbatim}

\begin{verbatim}
## Reading 011.csv
\end{verbatim}

\begin{verbatim}
## Rows: 860 Columns: 17
\end{verbatim}

\begin{verbatim}
## -- Column specification --------------------------------------------------------
## Delimiter: ","
## chr  (5): trial.type, task, target.type, obj.resp, subject
## dbl (12): pas, trial, jitter.x, jitter.y, odd.digit, target.contrast, target...
\end{verbatim}

\begin{verbatim}
## 
## i Use `spec()` to retrieve the full column specification for this data.
## i Specify the column types or set `show_col_types = FALSE` to quiet this message.
\end{verbatim}

\begin{verbatim}
## Reading 012.csv
\end{verbatim}

\begin{verbatim}
## Rows: 882 Columns: 17
\end{verbatim}

\begin{verbatim}
## -- Column specification --------------------------------------------------------
## Delimiter: ","
## chr  (5): trial.type, task, target.type, obj.resp, subject
## dbl (12): pas, trial, jitter.x, jitter.y, odd.digit, target.contrast, target...
\end{verbatim}

\begin{verbatim}
## 
## i Use `spec()` to retrieve the full column specification for this data.
## i Specify the column types or set `show_col_types = FALSE` to quiet this message.
\end{verbatim}

\begin{verbatim}
## Reading 013.csv
\end{verbatim}

\begin{verbatim}
## Rows: 882 Columns: 17
\end{verbatim}

\begin{verbatim}
## -- Column specification --------------------------------------------------------
## Delimiter: ","
## chr  (5): trial.type, task, target.type, obj.resp, subject
## dbl (12): pas, trial, jitter.x, jitter.y, odd.digit, target.contrast, target...
\end{verbatim}

\begin{verbatim}
## 
## i Use `spec()` to retrieve the full column specification for this data.
## i Specify the column types or set `show_col_types = FALSE` to quiet this message.
\end{verbatim}

\begin{verbatim}
## Reading 014.csv
\end{verbatim}

\begin{verbatim}
## Rows: 882 Columns: 17
\end{verbatim}

\begin{verbatim}
## -- Column specification --------------------------------------------------------
## Delimiter: ","
## chr  (5): trial.type, task, target.type, obj.resp, subject
## dbl (12): pas, trial, jitter.x, jitter.y, odd.digit, target.contrast, target...
\end{verbatim}

\begin{verbatim}
## 
## i Use `spec()` to retrieve the full column specification for this data.
## i Specify the column types or set `show_col_types = FALSE` to quiet this message.
\end{verbatim}

\begin{verbatim}
## Reading 015.csv
\end{verbatim}

\begin{verbatim}
## Rows: 882 Columns: 16
\end{verbatim}

\begin{verbatim}
## -- Column specification --------------------------------------------------------
## Delimiter: ","
## chr  (5): trial.type, task, target.type, obj.resp, subject
## dbl (11): pas, trial, jitter.x, jitter.y, odd.digit, target.contrast, target...
\end{verbatim}

\begin{verbatim}
## 
## i Use `spec()` to retrieve the full column specification for this data.
## i Specify the column types or set `show_col_types = FALSE` to quiet this message.
\end{verbatim}

\begin{verbatim}
## Reading 016.csv
\end{verbatim}

\begin{verbatim}
## Rows: 882 Columns: 17
\end{verbatim}

\begin{verbatim}
## -- Column specification --------------------------------------------------------
## Delimiter: ","
## chr  (5): trial.type, task, target.type, obj.resp, subject
## dbl (12): pas, trial, jitter.x, jitter.y, odd.digit, target.contrast, target...
\end{verbatim}

\begin{verbatim}
## 
## i Use `spec()` to retrieve the full column specification for this data.
## i Specify the column types or set `show_col_types = FALSE` to quiet this message.
\end{verbatim}

\begin{verbatim}
## Reading 017.csv
\end{verbatim}

\begin{verbatim}
## Rows: 882 Columns: 17
\end{verbatim}

\begin{verbatim}
## -- Column specification --------------------------------------------------------
## Delimiter: ","
## chr  (5): trial.type, task, target.type, obj.resp, subject
## dbl (12): pas, trial, jitter.x, jitter.y, odd.digit, target.contrast, target...
\end{verbatim}

\begin{verbatim}
## 
## i Use `spec()` to retrieve the full column specification for this data.
## i Specify the column types or set `show_col_types = FALSE` to quiet this message.
\end{verbatim}

\begin{verbatim}
## Reading 018.csv
\end{verbatim}

\begin{verbatim}
## Rows: 882 Columns: 17
\end{verbatim}

\begin{verbatim}
## -- Column specification --------------------------------------------------------
## Delimiter: ","
## chr  (5): trial.type, task, target.type, obj.resp, subject
## dbl (12): pas, trial, jitter.x, jitter.y, odd.digit, target.contrast, target...
\end{verbatim}

\begin{verbatim}
## 
## i Use `spec()` to retrieve the full column specification for this data.
## i Specify the column types or set `show_col_types = FALSE` to quiet this message.
\end{verbatim}

\begin{verbatim}
## Reading 019.csv
\end{verbatim}

\begin{verbatim}
## Rows: 882 Columns: 17
\end{verbatim}

\begin{verbatim}
## -- Column specification --------------------------------------------------------
## Delimiter: ","
## chr  (5): trial.type, task, target.type, obj.resp, subject
## dbl (12): pas, trial, jitter.x, jitter.y, odd.digit, target.contrast, target...
\end{verbatim}

\begin{verbatim}
## 
## i Use `spec()` to retrieve the full column specification for this data.
## i Specify the column types or set `show_col_types = FALSE` to quiet this message.
\end{verbatim}

\begin{verbatim}
## Reading 020.csv
\end{verbatim}

\begin{verbatim}
## Rows: 882 Columns: 17
\end{verbatim}

\begin{verbatim}
## -- Column specification --------------------------------------------------------
## Delimiter: ","
## chr  (5): trial.type, task, target.type, obj.resp, subject
## dbl (12): pas, trial, jitter.x, jitter.y, odd.digit, target.contrast, target...
\end{verbatim}

\begin{verbatim}
## 
## i Use `spec()` to retrieve the full column specification for this data.
## i Specify the column types or set `show_col_types = FALSE` to quiet this message.
\end{verbatim}

\begin{verbatim}
## Reading 021.csv
\end{verbatim}

\begin{verbatim}
## Rows: 882 Columns: 17
\end{verbatim}

\begin{verbatim}
## -- Column specification --------------------------------------------------------
## Delimiter: ","
## chr  (5): trial.type, task, target.type, obj.resp, subject
## dbl (12): pas, trial, jitter.x, jitter.y, odd.digit, target.contrast, target...
\end{verbatim}

\begin{verbatim}
## 
## i Use `spec()` to retrieve the full column specification for this data.
## i Specify the column types or set `show_col_types = FALSE` to quiet this message.
\end{verbatim}

\begin{verbatim}
## Reading 022.csv
\end{verbatim}

\begin{verbatim}
## Rows: 882 Columns: 17
\end{verbatim}

\begin{verbatim}
## -- Column specification --------------------------------------------------------
## Delimiter: ","
## chr  (5): trial.type, task, target.type, obj.resp, subject
## dbl (12): pas, trial, jitter.x, jitter.y, odd.digit, target.contrast, target...
\end{verbatim}

\begin{verbatim}
## 
## i Use `spec()` to retrieve the full column specification for this data.
## i Specify the column types or set `show_col_types = FALSE` to quiet this message.
\end{verbatim}

\begin{verbatim}
## Reading 023.csv
\end{verbatim}

\begin{verbatim}
## Rows: 882 Columns: 17
\end{verbatim}

\begin{verbatim}
## -- Column specification --------------------------------------------------------
## Delimiter: ","
## chr  (5): trial.type, task, target.type, obj.resp, subject
## dbl (12): pas, trial, jitter.x, jitter.y, odd.digit, target.contrast, target...
\end{verbatim}

\begin{verbatim}
## 
## i Use `spec()` to retrieve the full column specification for this data.
## i Specify the column types or set `show_col_types = FALSE` to quiet this message.
\end{verbatim}

\begin{verbatim}
## Reading 024.csv
\end{verbatim}

\begin{verbatim}
## Rows: 910 Columns: 17
\end{verbatim}

\begin{verbatim}
## -- Column specification --------------------------------------------------------
## Delimiter: ","
## chr  (5): trial.type, task, target.type, obj.resp, subject
## dbl (12): pas, trial, jitter.x, jitter.y, odd.digit, target.contrast, target...
\end{verbatim}

\begin{verbatim}
## 
## i Use `spec()` to retrieve the full column specification for this data.
## i Specify the column types or set `show_col_types = FALSE` to quiet this message.
\end{verbatim}

\begin{verbatim}
## Reading 025.csv
\end{verbatim}

\begin{verbatim}
## Rows: 900 Columns: 17
\end{verbatim}

\begin{verbatim}
## -- Column specification --------------------------------------------------------
## Delimiter: ","
## chr  (5): trial.type, task, target.type, obj.resp, subject
## dbl (12): pas, trial, jitter.x, jitter.y, odd.digit, target.contrast, target...
\end{verbatim}

\begin{verbatim}
## 
## i Use `spec()` to retrieve the full column specification for this data.
## i Specify the column types or set `show_col_types = FALSE` to quiet this message.
\end{verbatim}

\begin{verbatim}
## Reading 026.csv
\end{verbatim}

\begin{verbatim}
## Rows: 882 Columns: 17
\end{verbatim}

\begin{verbatim}
## -- Column specification --------------------------------------------------------
## Delimiter: ","
## chr  (5): trial.type, task, target.type, obj.resp, subject
## dbl (12): pas, trial, jitter.x, jitter.y, odd.digit, target.contrast, target...
\end{verbatim}

\begin{verbatim}
## 
## i Use `spec()` to retrieve the full column specification for this data.
## i Specify the column types or set `show_col_types = FALSE` to quiet this message.
\end{verbatim}

\begin{verbatim}
## Reading 027.csv
\end{verbatim}

\begin{verbatim}
## Rows: 882 Columns: 17
\end{verbatim}

\begin{verbatim}
## -- Column specification --------------------------------------------------------
## Delimiter: ","
## chr  (5): trial.type, task, target.type, obj.resp, subject
## dbl (12): pas, trial, jitter.x, jitter.y, odd.digit, target.contrast, target...
\end{verbatim}

\begin{verbatim}
## 
## i Use `spec()` to retrieve the full column specification for this data.
## i Specify the column types or set `show_col_types = FALSE` to quiet this message.
\end{verbatim}

\begin{verbatim}
## Reading 028.csv
\end{verbatim}

\begin{verbatim}
## Rows: 882 Columns: 17
\end{verbatim}

\begin{verbatim}
## -- Column specification --------------------------------------------------------
## Delimiter: ","
## chr  (5): trial.type, task, target.type, obj.resp, subject
## dbl (12): pas, trial, jitter.x, jitter.y, odd.digit, target.contrast, target...
\end{verbatim}

\begin{verbatim}
## 
## i Use `spec()` to retrieve the full column specification for this data.
## i Specify the column types or set `show_col_types = FALSE` to quiet this message.
\end{verbatim}

\begin{verbatim}
## Reading 029.csv
\end{verbatim}

\begin{verbatim}
## Rows: 882 Columns: 16
\end{verbatim}

\begin{verbatim}
## -- Column specification --------------------------------------------------------
## Delimiter: ","
## chr  (5): trial.type, task, target.type, obj.resp, subject
## dbl (11): pas, trial, jitter.x, jitter.y, odd.digit, target.contrast, target...
\end{verbatim}

\begin{verbatim}
## 
## i Use `spec()` to retrieve the full column specification for this data.
## i Specify the column types or set `show_col_types = FALSE` to quiet this message.
\end{verbatim}

\begin{Shaded}
\begin{Highlighting}[]
\NormalTok{df\_2 }\OtherTok{=}\NormalTok{ df\_2 }\SpecialCharTok{\%\textgreater{}\%} 
  \FunctionTok{filter}\NormalTok{(trial.type }\SpecialCharTok{!=} \StringTok{"practice"}\NormalTok{)}

\NormalTok{df\_2 }\OtherTok{=}\NormalTok{ df\_2 }\SpecialCharTok{\%\textgreater{}\%} 
  \FunctionTok{mutate}\NormalTok{(}\AttributeTok{correct =} \FunctionTok{ifelse}\NormalTok{(target.type }\SpecialCharTok{==} \StringTok{"odd"} \SpecialCharTok{\&}\NormalTok{ obj.resp }\SpecialCharTok{==} \StringTok{"o"}\NormalTok{,}\DecValTok{1}\NormalTok{,}
                          \FunctionTok{ifelse}\NormalTok{(target.type }\SpecialCharTok{==} \StringTok{"even"} \SpecialCharTok{\&}\NormalTok{ obj.resp }\SpecialCharTok{==} \StringTok{"e"}\NormalTok{,}\DecValTok{1}\NormalTok{,}\DecValTok{0}\NormalTok{)))}

\NormalTok{df\_2}\SpecialCharTok{$}\NormalTok{correct }\OtherTok{=} \FunctionTok{as.logical}\NormalTok{(df\_2}\SpecialCharTok{$}\NormalTok{correct)}

\FunctionTok{class}\NormalTok{(df\_2}\SpecialCharTok{$}\NormalTok{correct)}
\end{Highlighting}
\end{Shaded}

\begin{verbatim}
## [1] "logical"
\end{verbatim}

\begin{Shaded}
\begin{Highlighting}[]
\NormalTok{log\_model\_partial\_pool\_2 }\OtherTok{=} \FunctionTok{glmer}\NormalTok{(correct }\SpecialCharTok{\textasciitilde{}}\NormalTok{ target.frames }\SpecialCharTok{+}\NormalTok{ (}\DecValTok{1}\SpecialCharTok{|}\NormalTok{subject), }\AttributeTok{data =}\NormalTok{ df\_2, }\AttributeTok{family =} \FunctionTok{binomial}\NormalTok{(}\AttributeTok{link =} \StringTok{"logit"}\NormalTok{))}



\FunctionTok{summary}\NormalTok{(log\_model\_partial\_pool);}\FunctionTok{summary}\NormalTok{(log\_model\_partial\_pool\_2)}
\end{Highlighting}
\end{Shaded}

\begin{verbatim}
## Generalized linear mixed model fit by maximum likelihood (Laplace
##   Approximation) [glmerMod]
##  Family: binomial  ( logit )
## Formula: correct ~ target.frames + (1 | subject)
##    Data: df
## 
##      AIC      BIC   logLik deviance df.resid 
##  21250.1  21274.4 -10622.0  21244.1    25041 
## 
## Scaled residuals: 
##     Min      1Q  Median      3Q     Max 
## -7.7520  0.1436  0.2604  0.4816  2.0730 
## 
## Random effects:
##  Groups  Name        Variance Std.Dev.
##  subject (Intercept) 0.1549   0.3936  
## Number of obs: 25044, groups:  subject, 29
## 
## Fixed effects:
##               Estimate Std. Error z value Pr(>|z|)    
## (Intercept)   -0.95968    0.08128  -11.81   <2e-16 ***
## target.frames  0.75493    0.01251   60.36   <2e-16 ***
## ---
## Signif. codes:  0 '***' 0.001 '**' 0.01 '*' 0.05 '.' 0.1 ' ' 1
## 
## Correlation of Fixed Effects:
##             (Intr)
## target.frms -0.382
\end{verbatim}

\begin{verbatim}
## Generalized linear mixed model fit by maximum likelihood (Laplace
##   Approximation) [glmerMod]
##  Family: binomial  ( logit )
## Formula: correct ~ target.frames + (1 | subject)
##    Data: df_2
## 
##      AIC      BIC   logLik deviance df.resid 
##  21250.1  21274.4 -10622.0  21244.1    25041 
## 
## Scaled residuals: 
##     Min      1Q  Median      3Q     Max 
## -7.7520  0.1436  0.2604  0.4816  2.0730 
## 
## Random effects:
##  Groups  Name        Variance Std.Dev.
##  subject (Intercept) 0.1549   0.3936  
## Number of obs: 25044, groups:  subject, 29
## 
## Fixed effects:
##               Estimate Std. Error z value Pr(>|z|)    
## (Intercept)   -0.95968    0.08128  -11.81   <2e-16 ***
## target.frames  0.75493    0.01251   60.36   <2e-16 ***
## ---
## Signif. codes:  0 '***' 0.001 '**' 0.01 '*' 0.05 '.' 0.1 ' ' 1
## 
## Correlation of Fixed Effects:
##             (Intr)
## target.frms -0.382
\end{verbatim}

\begin{enumerate}
\def\labelenumi{\roman{enumi}.}
\tightlist
\item
  the likelihood-function for logistic regression is:
  \(L(p)={\displaystyle\prod_{i=1}^Np^{y_i}(1-p)^{(1-y_i)}}\) (Remember
  the probability mass function for the Bernoulli Distribution). Create
  a function that calculates the likelihood.
\end{enumerate}

\begin{Shaded}
\begin{Highlighting}[]
\NormalTok{likelihood }\OtherTok{\textless{}{-}} \ControlFlowTok{function}\NormalTok{(i)\{}
\NormalTok{  fitted\_values }\OtherTok{=} \FunctionTok{fitted}\NormalTok{(i)}
\NormalTok{  y\_values }\OtherTok{=} \FunctionTok{as.vector}\NormalTok{(}\FunctionTok{model.response}\NormalTok{(}\FunctionTok{model.frame}\NormalTok{(i), }\AttributeTok{type =} \StringTok{"numeric"}\NormalTok{))}
\NormalTok{  likelihood }\OtherTok{=} \FunctionTok{prod}\NormalTok{((fitted\_values}\SpecialCharTok{\^{}}\NormalTok{y\_values)}\SpecialCharTok{*}\NormalTok{(}\DecValTok{1}\SpecialCharTok{{-}}\NormalTok{fitted\_values)}\SpecialCharTok{\^{}}\NormalTok{(}\DecValTok{1}\SpecialCharTok{{-}}\NormalTok{y\_values))}
  \FunctionTok{return}\NormalTok{(likelihood)}
\NormalTok{\}}
\end{Highlighting}
\end{Shaded}

\begin{enumerate}
\def\labelenumi{\roman{enumi}.}
\setcounter{enumi}{1}
\tightlist
\item
  the log-likelihood-function for logistic regression is:
  \(l(p) = {\displaystyle\sum_{i=1}^N}[y_i\ln{p}+(1-y_i)\ln{(1-p)}\).
  Create a function that calculates the log-likelihood
\end{enumerate}

\begin{Shaded}
\begin{Highlighting}[]
\NormalTok{log\_likelihood }\OtherTok{\textless{}{-}} \ControlFlowTok{function}\NormalTok{(i)\{}
\NormalTok{  fitted\_values }\OtherTok{=} \FunctionTok{fitted}\NormalTok{(i)}
\NormalTok{  y\_values }\OtherTok{=} \FunctionTok{as.vector}\NormalTok{(}\FunctionTok{model.response}\NormalTok{(}\FunctionTok{model.frame}\NormalTok{(i), }\AttributeTok{type =} \StringTok{"numeric"}\NormalTok{))}
\NormalTok{  log\_likelihood }\OtherTok{=} \FunctionTok{sum}\NormalTok{(y\_values}\SpecialCharTok{*}\FunctionTok{log}\NormalTok{(fitted\_values)}\SpecialCharTok{+}\NormalTok{(}\DecValTok{1}\SpecialCharTok{{-}}\NormalTok{y\_values)}\SpecialCharTok{*}\FunctionTok{log}\NormalTok{(}\DecValTok{1} \SpecialCharTok{{-}}\NormalTok{ fitted\_values))}
  \FunctionTok{return}\NormalTok{(log\_likelihood)}
\NormalTok{\}}
\end{Highlighting}
\end{Shaded}

\begin{enumerate}
\def\labelenumi{\roman{enumi}.}
\setcounter{enumi}{2}
\tightlist
\item
  apply both functions to the pooling model you just created. Make sure
  that the log-likelihood matches what is returned from the
  \emph{logLik} function for the pooled model. Does the
  likelihood-function return a value that is surprising? Why is the
  log-likelihood preferable when working with computers with limited
  precision?
\end{enumerate}

\begin{Shaded}
\begin{Highlighting}[]
\FunctionTok{likelihood}\NormalTok{(log\_model\_pool)}
\end{Highlighting}
\end{Shaded}

\begin{verbatim}
## [1] 0
\end{verbatim}

\begin{Shaded}
\begin{Highlighting}[]
\FunctionTok{format}\NormalTok{(}\FunctionTok{likelihood}\NormalTok{(log\_model\_pool), }\AttributeTok{scientific =} \ConstantTok{TRUE}\NormalTok{) }\CommentTok{\#Try to format the output in scientific notation, to see if the output of 0 has something to do with rounding.}
\end{Highlighting}
\end{Shaded}

\begin{verbatim}
## [1] "0e+00"
\end{verbatim}

\begin{Shaded}
\begin{Highlighting}[]
\FunctionTok{log\_likelihood}\NormalTok{(log\_model\_pool)}
\end{Highlighting}
\end{Shaded}

\begin{verbatim}
## [1] -10865.25
\end{verbatim}

\begin{Shaded}
\begin{Highlighting}[]
\FunctionTok{logLik}\NormalTok{(log\_model\_pool)}
\end{Highlighting}
\end{Shaded}

\begin{verbatim}
## 'log Lik.' -10865.25 (df=2)
\end{verbatim}

Our the output of our hand-made log\_likelihood function matches the one
of the logLik function, so we're happy. The output from our constructed
likelihood function is a bit surprising, since it just gives the output
zero - I even though when we try to format the output to scientific
notation, since the 0 might be due to rounding done by R, but it also
results in 0. This is probably due to the value of the output of the
likelihood function is extremely small, since it is 25044 probabilities
(i.e.~a number between 0-1) multiplied together, and thus being so
small, that a computer with limited precision as my own rounds the
output to 0. Because of that, it is preferable to take the
log-likelihood instead, since it returns a number that the computer can
actually capture and doesn't get lost to limited precision.

\begin{enumerate}
\def\labelenumi{\roman{enumi}.}
\setcounter{enumi}{3}
\tightlist
\item
  now show that the log-likelihood is a little off when applied to the
  partial pooling model - (the likelihood function is different for the
  multilevel function - see section 2.1 of
  \url{https://www.researchgate.net/profile/Douglas-Bates/publication/2753537_Computational_Methods_for_Multilevel_Modelling/links/00b4953b4108d73427000000/Computational-Methods-for-Multilevel-Modelling.pdf}
  if you are interested)
\end{enumerate}

\begin{Shaded}
\begin{Highlighting}[]
\FunctionTok{log\_likelihood}\NormalTok{(log\_model\_partial\_pool)}
\end{Highlighting}
\end{Shaded}

\begin{verbatim}
## [1] -10565.53
\end{verbatim}

\begin{Shaded}
\begin{Highlighting}[]
\FunctionTok{logLik}\NormalTok{(log\_model\_partial\_pool)}
\end{Highlighting}
\end{Shaded}

\begin{verbatim}
## 'log Lik.' -10622.03 (df=3)
\end{verbatim}

We se, that the two results differ from each other - the way to
calculate the log-likelihood of a pooled logistic regression model is
therefore different from the calculating the log-likelihood of a partial
pooled model

\begin{Shaded}
\begin{Highlighting}[]
\NormalTok{null\_log\_model }\OtherTok{=} \FunctionTok{glm}\NormalTok{(correct }\SpecialCharTok{\textasciitilde{}} \DecValTok{1}\NormalTok{, }\AttributeTok{data =}\NormalTok{ df, }\AttributeTok{family =} \FunctionTok{binomial}\NormalTok{(}\AttributeTok{link =} \StringTok{"logit"}\NormalTok{))}

\NormalTok{log\_model\_pool }\OtherTok{=} \FunctionTok{glm}\NormalTok{(correct }\SpecialCharTok{\textasciitilde{}}\NormalTok{ target.frames, }\AttributeTok{data =}\NormalTok{ df, }\AttributeTok{family =} \FunctionTok{binomial}\NormalTok{(}\AttributeTok{link =} \StringTok{"logit"}\NormalTok{))}

\NormalTok{log\_model\_partial\_pool }\OtherTok{=} \FunctionTok{glmer}\NormalTok{(correct }\SpecialCharTok{\textasciitilde{}}\NormalTok{ target.frames }\SpecialCharTok{+}\NormalTok{ (}\DecValTok{1}\SpecialCharTok{|}\NormalTok{subject), }\AttributeTok{data =}\NormalTok{ df, }\AttributeTok{family =} \FunctionTok{binomial}\NormalTok{(}\AttributeTok{link =} \StringTok{"logit"}\NormalTok{))}
\end{Highlighting}
\end{Shaded}

\begin{enumerate}
\def\labelenumi{\arabic{enumi})}
\setcounter{enumi}{1}
\tightlist
\item
  Use log-likelihood ratio tests to argue for the addition of predictor
  variables, start from the null model,
  \texttt{glm(correct\ \textasciitilde{}\ 1,\ \textquotesingle{}binomial\textquotesingle{},\ data)},
  then add subject-level intercepts, then add a group-level effect of
  \emph{target.frames} and finally add subject-level slopes for
  \emph{target.frames}. Also assess whether or not a correlation between
  the subject-level slopes and the subject-level intercepts should be
  included.

  \begin{enumerate}
  \def\labelenumii{\roman{enumii}.}
  \tightlist
  \item
    write a short methods section and a results section where you
    indicate which model you chose and the statistics relevant for that
    choice. Include a plot of the estimated group-level function with
    \texttt{xlim=c(0,\ 8)} that includes the estimated subject-specific
    functions.
  \end{enumerate}
\end{enumerate}

\begin{enumerate}
\def\labelenumi{\roman{enumi}.}
\setcounter{enumi}{1}
\tightlist
\item
  also include in the results section whether the fit didn't look good
  for any of the subjects. If so, identify those subjects in the report,
  and judge (no statistical test) whether their performance (accuracy)
  differed from that of the other subjects. Was their performance better
  than chance? (Use a statistical test this time) (50 \%)
\end{enumerate}

\begin{enumerate}
\def\labelenumi{\arabic{enumi})}
\setcounter{enumi}{2}
\tightlist
\item
  Now add \emph{pas} to the group-level effects - if a log-likelihood
  ratio test justifies this, also add the interaction between \emph{pas}
  and \emph{target.frames} and check whether a log-likelihood ratio test
  justifies this
\end{enumerate}

\begin{enumerate}
\def\labelenumi{\roman{enumi}.}
\item
  if your model doesn't converge, try a different optimizer
\item
  plot the estimated group-level functions over \texttt{xlim=c(0,\ 8)}
  for each of the four PAS-ratings - add this plot to your report (see:
  5.2.i) and add a description of your chosen model. Describe how
  \emph{pas} affects accuracy together with target duration if at all.
  Also comment on the estimated functions' behaviour at target.frame=0 -
  is that behaviour reasonable?
\end{enumerate}

\hypertarget{exercise-6---test-linear-hypotheses}{%
\section{EXERCISE 6 - Test linear
hypotheses}\label{exercise-6---test-linear-hypotheses}}

In this section we are going to test different hypotheses. We assume
that we have already proved that more objective evidence (longer
duration of stimuli) is sufficient to increase accuracy in and of itself
and that more subjective evidence (higher PAS ratings) is also
sufficient to increase accuracy in and of itself.\\
We want to test a hypothesis for each of the three neighbouring
differences in PAS, i.e.~the difference between 2 and 1, the difference
between 3 and 2 and the difference between 4 and 3. More specifically,
we want to test the hypothesis that accuracy increases faster with
objective evidence if subjective evidence is higher at the same time,
i.e.~we want to test for an interaction.

\begin{enumerate}
\def\labelenumi{\arabic{enumi})}
\tightlist
\item
  Fit a model based on the following formula:
  \texttt{correct\ \textasciitilde{}\ pas\ *\ target.frames\ +\ (target.frames\ \textbar{}\ subject))}

  \begin{enumerate}
  \def\labelenumii{\roman{enumii}.}
  \tightlist
  \item
    First, use \texttt{summary} (yes, you are allowed to!) to argue that
    accuracy increases faster with objective evidence for PAS 2 than for
    PAS 1.
  \end{enumerate}
\item
  \texttt{summary} won't allow you to test whether accuracy increases
  faster with objective evidence for PAS 3 than for PAS 2 (unless you
  use \texttt{relevel}, which you are not allowed to in this exercise).
  Instead, we'll be using the function \texttt{glht} from the
  \texttt{multcomp} package

  \begin{enumerate}
  \def\labelenumii{\roman{enumii}.}
  \tightlist
  \item
    To redo the test in 6.1.i, you can create a \emph{contrast} vector.
    This vector will have the length of the number of estimated
    group-level effects and any specific contrast you can think of can
    be specified using this. For redoing the test from 6.1.i, the code
    snippet below will do
  \item
    Now test the hypothesis that accuracy increases faster with
    objective evidence for PAS 3 than for PAS 2.
  \item
    Also test the hypothesis that accuracy increases faster with
    objective evidence for PAS 4 than for PAS 3
  \end{enumerate}
\item
  Finally, test that whether the difference between PAS 2 and 1 (tested
  in 6.1.i) is greater than the difference between PAS 4 and 3 (tested
  in 6.2.iii)
\end{enumerate}

\hypertarget{snippet-for-6.2.i}{%
\subsubsection{Snippet for 6.2.i}\label{snippet-for-6.2.i}}

\begin{Shaded}
\begin{Highlighting}[]
\DocumentationTok{\#\# testing whether PAS 2 is different from PAS 1}
\NormalTok{contrast.vector }\OtherTok{\textless{}{-}} \FunctionTok{matrix}\NormalTok{(}\FunctionTok{c}\NormalTok{(}\DecValTok{0}\NormalTok{, }\DecValTok{0}\NormalTok{, }\DecValTok{0}\NormalTok{, }\DecValTok{0}\NormalTok{, }\DecValTok{0}\NormalTok{, }\DecValTok{1}\NormalTok{, }\DecValTok{0}\NormalTok{, }\DecValTok{0}\NormalTok{), }\AttributeTok{nrow=}\DecValTok{1}\NormalTok{)}
\NormalTok{gh }\OtherTok{\textless{}{-}} \FunctionTok{glht}\NormalTok{(pas.intact.tf.ranslopeint.with.corr, contrast.vector)}
\FunctionTok{print}\NormalTok{(}\FunctionTok{summary}\NormalTok{(gh))}
\DocumentationTok{\#\# as another example, we could also test whether there is a difference in}
\DocumentationTok{\#\# intercepts between PAS 2 and PAS 3}
\NormalTok{contrast.vector }\OtherTok{\textless{}{-}} \FunctionTok{matrix}\NormalTok{(}\FunctionTok{c}\NormalTok{(}\DecValTok{0}\NormalTok{, }\SpecialCharTok{{-}}\DecValTok{1}\NormalTok{, }\DecValTok{1}\NormalTok{, }\DecValTok{0}\NormalTok{, }\DecValTok{0}\NormalTok{, }\DecValTok{0}\NormalTok{, }\DecValTok{0}\NormalTok{, }\DecValTok{0}\NormalTok{), }\AttributeTok{nrow=}\DecValTok{1}\NormalTok{)}
\NormalTok{gh }\OtherTok{\textless{}{-}} \FunctionTok{glht}\NormalTok{(pas.intact.tf.ranslopeint.with.corr, contrast.vector)}
\FunctionTok{print}\NormalTok{(}\FunctionTok{summary}\NormalTok{(gh))}
\end{Highlighting}
\end{Shaded}

\hypertarget{exercise-7---estimate-psychometric-functions-for-the-perceptual-awareness-scale-and-evaluate-them}{%
\section{EXERCISE 7 - Estimate psychometric functions for the Perceptual
Awareness Scale and evaluate
them}\label{exercise-7---estimate-psychometric-functions-for-the-perceptual-awareness-scale-and-evaluate-them}}

We saw in 5.3 that the estimated functions went below chance at a target
duration of 0 frames (0 ms). This does not seem reasonable, so we will
be trying a different approach for fitting here.\\
We will fit the following function that results in a sigmoid,
\(f(x) = a + \frac {b - a} {1 + e^{\frac {c-x} {d}}}\)\\
It has four parameters: \emph{a}, which can be interpreted as the
minimum accuracy level, \emph{b}, which can be interpreted as the
maximum accuracy level, \emph{c}, which can be interpreted as the
so-called inflexion point, i.e.~where the derivative of the sigmoid
reaches its maximum and \emph{d}, which can be interpreted as the
steepness at the inflexion point. (When \emph{d} goes towards infinity,
the slope goes towards a straight line, and when it goes towards 0, the
slope goes towards a step function).

We can define a function of a residual sum of squares as below

\begin{Shaded}
\begin{Highlighting}[]
\NormalTok{RSS }\OtherTok{\textless{}{-}} \ControlFlowTok{function}\NormalTok{(dataset, par)}
\NormalTok{\{}
    \DocumentationTok{\#\# "dataset" should be a data.frame containing the variables x (target.frames)}
    \DocumentationTok{\#\# and y (correct)}
    
    \DocumentationTok{\#\# "par" are our four parameters (a numeric vector) }
    \DocumentationTok{\#\# par[1]=a, par[2]=b, par[3]=c, par[4]=d}
\NormalTok{    x }\OtherTok{\textless{}{-}}\NormalTok{ dataset}\SpecialCharTok{$}\NormalTok{x}
\NormalTok{    y }\OtherTok{\textless{}{-}}\NormalTok{ dataset}\SpecialCharTok{$}\NormalTok{y}
\NormalTok{    y.hat }\OtherTok{\textless{}{-}} \DocumentationTok{\#\# you fill in the estimate of y.hat}
\NormalTok{    RSS }\OtherTok{\textless{}{-}} \FunctionTok{sum}\NormalTok{((y }\SpecialCharTok{{-}}\NormalTok{ y.hat)}\SpecialCharTok{\^{}}\DecValTok{2}\NormalTok{)}
    \FunctionTok{return}\NormalTok{(RSS)}
\NormalTok{\}}
\end{Highlighting}
\end{Shaded}

\begin{enumerate}
\def\labelenumi{\arabic{enumi})}
\tightlist
\item
  Now, we will fit the sigmoid for the four PAS ratings for Subject 7

  \begin{enumerate}
  \def\labelenumii{\roman{enumii}.}
  \tightlist
  \item
    use the function \texttt{optim}. It returns a list that among other
    things contains the four estimated parameters. You should set the
    following arguments:\\
    \texttt{par}: you can set \emph{c} and \emph{d} as 1. Find good
    choices for \emph{a} and \emph{b} yourself (and argue why they are
    appropriate)\\
    \texttt{fn}: which function to minimise?\\
    \texttt{data}: the data frame with \emph{x}, \emph{target.frames},
    and \emph{y}, \emph{correct} in it\\
    \texttt{method}: `L-BFGS-B'\\
    \texttt{lower}: lower bounds for the four parameters, (the lowest
    value they can take), you can set \emph{c} and \emph{d} as
    \texttt{-Inf}. Find good choices for \emph{a} and \emph{b} yourself
    (and argue why they are appropriate)\\
    \texttt{upper}: upper bounds for the four parameters, (the highest
    value they can take) can set \emph{c} and \emph{d} as \texttt{Inf}.
    Find good choices for \emph{a} and \emph{b} yourself (and argue why
    they are appropriate)\\
  \item
    Plot the fits for the PAS ratings on a single plot (for subject 7)
    \texttt{xlim=c(0,\ 8)}
  \item
    Create a similar plot for the PAS ratings on a single plot (for
    subject 7), but this time based on the model from 6.1
    \texttt{xlim=c(0,\ 8)}\\
  \item
    Comment on the differences between the fits - mention some
    advantages and disadvantages of each way\\
  \end{enumerate}
\item
  Finally, estimate the parameters for all subjects and each of their
  four PAS ratings. Then plot the estimated function at the group-level
  by taking the mean for each of the four parameters, \emph{a},
  \emph{b}, \emph{c} and \emph{d} across subjects. A function should be
  estimated for each PAS-rating (it should look somewhat similar to Fig.
  3 from the article:
  \url{https://doi.org/10.1016/j.concog.2019.03.007})

  \begin{enumerate}
  \def\labelenumii{\roman{enumii}.}
  \tightlist
  \item
    compare with the figure you made in 5.3.ii and comment on the
    differences between the fits - mention some advantages and
    disadvantages of both.
  \end{enumerate}
\end{enumerate}

\end{document}
